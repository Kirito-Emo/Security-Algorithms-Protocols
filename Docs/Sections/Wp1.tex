\chapter{WORK PACKAGE 1}
In questo primo capitolo ci concentreremo sulla definizione degli attori onesti coinvolti nel sistema, analizzando i loro obiettivi e le funzionalità che si intendono realizzare.
Gli attori onesti sono quegli individui o entità che agiscono in conformità con le regole e le politiche stabilite, cercando di raggiungere i loro obiettivi senza compromettere l'integrità del sistema.

Successivamente, esamineremo i possibili avversari (o threat models) che potrebbero essere interessati a compromettere il sistema, esaminando le loro risorse e le motivazioni che li spingono ad agire.
Questa analisi ci permetterà di identificare gli attacchi che il sistema potrebbe subire e di comprendere quali misure di sicurezza dovranno essere adottate per contrastarli.

Una volta compreso il contesto in cui il sistema opera, identificheremo le proprietà di sicurezza che si vorrebbero preservare in presenza di attacchi.
Queste proprietà sono fondamentali per garantire il corretto funzionamento del sistema e la protezione delle informazioni sensibili.

\noindent Di seguito un elenco dettagliato di tali attori e delle loro responsabilità.

\section{Attori del Sistema}

    \subsection{Utente}
        Gli utenti sono le persone che necessitano di accedere ai servizi qualificati offerti dai server.
        Ogni utente possiede una Carta d'Identità Elettronica (CIE) che contiene le credenziali rilasciate dalle autorità competenti.
        La CIE permette agli utenti di firmare digitalmente le loro richieste di accesso, garantendo così l'integrità e l'autenticità delle credenziali presentate.
         I loro obiettivi sono: richiedere e ottenere le credenziali necessarie dalle autorità, per poi utilizzarle per accedere a servizi specifici in modo sicuro; mentre il loro compito è quello di proteggere CIE e PIN da accessi non autorizzati.

    
    \subsection{Server}
        I server sono entità che offrono servizi qualificati, accessibili solo ai possessori di specifiche credenziali.
        I server devono essere in grado di verificare le credenziali presentate dagli utenti e stabilire se sono rilasciate da autorità fidate.
        Il loro obiettivo è offrire servizi in modo sicuro e limitato solo agli utenti autorizzati.
        Sono, inoltre, responsabili di gestire le richieste di accesso, garantire la sicurezza e la privacy delle operazioni e minimizzare il coinvolgimento di terze parti fidate così da ridurre i rischi legati ad un singolo punto di fallimento.

        
    \subsection{Autorità di Rilascio delle Credenziali}
        Le autorità di rilascio sono enti fidati che emettono credenziali agli utenti, garantendo che queste siano valide e affidabili.
        Queste autorità verificano l'identità degli utenti e le informazioni correlate prima di rilasciare le credenziali.
        Esse possono essere enti governativi, istituzioni pubbliche o altre organizzazioni autorizzate a rilasciare certificati digitali.
        Sono, quindi, responsabili di:
            \begin{itemize}
                \item verificare e validare le informazioni degli utenti;

                \item emettere credenziali autentiche;

                \item garantirne la sicurezza e l'integrità;

                \item mantenere un registro delle credenziali emesse.
            \end{itemize}


    \subsection{Autorità Rilascio CIE}
        Gli enti che si occupano dell'emissione della CIE con il PIN associato sono fondamentalmente due, l'ufficio anagrafe e l'Istituto Poligrafico e Zecca dello Stato (IPZS), che vengono accorpati in un'unica entità per convenienza e semplicità. Questa autorità si occupa della verifica dell'identità degli utenti, della gestione delle pratiche amministrative relative alla registrazione delle informazioni anagrafiche e della generazione e gestione delle chiavi crittografiche e dei certificati digitali necessari per la firma elettronica.
        È responsabile di:
            \begin{itemize}
                \item verificare l'identità degli utenti;

                \item registrare le informazioni anagrafiche;

                \item produrre e consegnare la CIE, insieme al suo PIN, all'utente;
            
                \item generare e memorizzare in modo sicuro le chiavi segrete e i certificati digitali all'interno della CIE;

                \item garantire che i certificati digitali siano validi e riconosciuti dalle autorità competenti;

                \item implementare misure di sicurezza avanzate per proteggere la CIE da contraffazioni e accessi non autorizzati.
            \end{itemize}


\section{Attaccanti del Sistema}
    Nel contesto del sistema descritto, è fondamentale analizzare e comprendere i potenziali attaccanti che potrebbero cercare di comprometterlo, considerando le loro motivazioni e le risorse a loro disposizione.
    Questi attori e gruppi di attaccanti rappresentano diverse minacce al sistema e richiedono misure di sicurezza specifiche per essere contrastati efficacemente.

    \begin{itemize}
        \item \textbf{Anti-Tech Theorist}
            \begin{itemize}
                \item \textbf{Tipologia:} passivo/attivo
                
                \item \textbf{Descrizione:} avversario che si oppone all’innovazione e all'adozione di nuove tecnologie, sostenendo che qualunque dispositivo può essere compromesso, a discapito della privacy degli utenti.
                Il suo scopo è quello di amplificare notizie di violazioni di dati o problemi di privacy dei sistemi per influenzare l’opinione pubblica e screditare le nuove proposte nell'ambito della sicurezza informatica.
               
                \item \textbf{Risorse}
                    \begin{itemize}
                        \item Abilità comunicative elevate

                        \item Rete di contatti con altri oppositori delle nuove tecnologie

                        \item Accesso ai media e ai social network per diffondere messaggi allarmistici
                    \end{itemize}
            \end{itemize}


        \item \textbf{The Eavesdropper}
            \begin{itemize}
                \item \textbf{Tipologia:} passivo
                
                \item \textbf{Descrizione:} avversario interessato a intercettare tutte le operazioni svolte tramite il sistema (autenticazione, dati della CIE, etc.).
                Questa tipologia di avversario può essere sia interna all’organizzazione che esterna.
                L'obiettivo principale è quello di collezionare le informazioni inerenti agli utenti, senza però compromettere i protocolli, i quali sono eseguiti onestamente.
                
                \item \textbf{Risorse}
                    \begin{itemize}
                        \item Conoscenza approfondita delle tecniche di intercettazione e analisi delle comunicazioni, nonché della crittografia utilizzata nel sistema per comprendere e interpretare correttamente i dati intercettati

                        \vspace{3mm}

                        \item Capacità di monitorare e intercettare le comunicazioni tra gli utenti e il sistema, compresi i messaggi di autenticazione, i dati della CIE e qualsiasi altra informazione scambiata

                        \vspace{3mm}

                        \item Disponibilità di risorse computazionali significative per analizzare e elaborare grandi quantità di dati intercettati
                    \end{itemize}
            \end{itemize}


        \item \textbf{Malevolent Server}
            \begin{itemize}
                \item \textbf{Tipologia:} attivo
                
                \item \textbf{Descrizione:} server con comportamento malevolo.
                I suoi compiti includono: la corretta gestione delle connessioni, la verifica delle credenziali e l'erogazione del servizio richiesto.
                Tuttavia, questa tipologia di avversario sfrutta le informazioni ricevuti dagli utenti per scopi illeciti, come il riutilizzo di informazioni personali, la vendita dei dati carpiti a terzi, compromettendo la privacy degli utenti.
                
                \item \textbf{Risorse}
                    \begin{itemize}
                        \item Controllo completo sui dati e sulle risorse del server, inclusi database utenti, registri di accesso e altri dati sensibili

                        \vspace{3mm}

                        \item Possibilità di accedere alle credenziali ricevute per l'accesso al servizio, consentendo di utilizzare queste informazioni per scopi illeciti.
                    \end{itemize}
            \end{itemize}


        \item \textbf{The Impersonator}
            \begin{itemize}
                \item \textbf{Tipologia:} attivo
                
                \item \textbf{Descrizione:} avversario che, essendo entrato in possesso di informazioni personali di un utente, si maschera personificando quest’ultimo.
                Mira ad ottenere uno o più servizi a nome della persona di cui ha i dati.
                
                \item \textbf{Risorse}
                    \begin{itemize}
                        \item Possesso di un vasto database di informazioni personali ottenute illegalmente, che consente la creazione di profili dettagliati delle vittime e di utilizzare le informazioni per scopi malevoli
                    \end{itemize}
            \end{itemize}


        \item \textbf{Identity Thief}
            \begin{itemize}
                \item \textbf{Tipologia:} attivo

                \item \textbf{Descrizione:} avversario che mira a rubare identità o credenziali per impersonare gli utenti legittimi e ottenere accesso non autorizzato ai servizi.
                Potrebbe utilizzare tecniche di phishing, furto di credenziali o exploit delle vulnerabilità per ottenere informazioni sensibili e assumere l'identità di altri utenti.

                \item \textbf{Risorse}
                    \begin{itemize}
                        \item Disponibilità di risorse computazionali sufficienti per eseguire attacchi brute-force, decrittazione o altri metodi per ottenere informazioni sensibili o compromettere la sicurezza del sistema

                        \vspace{3mm}

                        \item Competenze nel phishing e nell'ingegneria sociale, accesso a database di credenziali rubate, capacità di creare siti web e messaggi convincenti
                    \end{itemize}
            \end{itemize}


        \item \textbf{The Denier}
            \begin{itemize}
                \item \textbf{Tipologia:} attivo
                
                \item \textbf{Descrizione:} avversario il cui scopo è compromettere il sistema, colpendo gli utenti o i servizi contemporaneamente.
                Potrebbe utilizzare botnets, malware o attacchi DoS/DDoS per sovraccaricare o infiltrare il sistema, causando danni diffusi e gravi interruzioni del servizio.
                
                \item \textbf{Risorse}
                    \begin{itemize}
                        \item Capacità di coordinare attacchi su vasta scala

                        \vspace{3mm}

                        \item Accesso a risorse informatiche distribuite (come server cloud, dispositivi IoT compromessi o altre infrastrutture che possono essere sfruttate per eseguire attacchi distribuiti)

                        \vspace{3mm}

                        \item Possibilità di avere motivazioni finanziarie o ideologiche significative per condurre gli attacchi, come il lucro finanziario, il furto di dati sensibili o il danneggiamento delle infrastrutture critiche per fini politici o di estorsione
                    \end{itemize}
            \end{itemize}


        \item \textbf{Evil Insider}
            \begin{itemize}
                \item \textbf{Tipologia:} passivo/attivo
                
                \item \textbf{Descrizione:} avversario interno al sistema, che agisce in modo malevolo per ottenere benefici personali o danneggiare l'organizzazione.
                Potrebbe rubare credenziali o informazioni sensibili, manipolare il sistema o sabotare le operazioni per raggiungere i suoi obiettivi.
                
                \item \textbf{Risorse}
                    \begin{itemize}
                        \item Autorizzazioni elevate all'interno del sistema, compreso l'accesso a dati sensibili e risorse critiche

                        \vspace{3mm}

                        \item Conoscenza interna dei processi e delle vulnerabilità
                    \end{itemize}
            \end{itemize}


        \item \textbf{Authority Hater}
            \begin{itemize}
                \item \textbf{Tipologia:} attivo
                
                \item \textbf{Descrizione:} avversario che mira a compromettere le autorità di certificazione o le infrastrutture chiave del sistema di identificazione, mettendo in dubbio l'affidabilità delle credenziali emesse o manipolando i certificati digitali.
                Potrebbe condurre attacchi di tipo man-in-the-middle, falsificare certificati o compromettere le chiavi crittografiche per ottenere accessi non autorizzati.
                
                \item \textbf{Risorse}
                    \begin{itemize}
                        \item Comprensione approfondita dei protocolli crittografici, dei meccanismi di autenticazione e delle vulnerabilità delle autorità di certificazione

                        \vspace{3mm}

                        \item Accesso a infrastrutture chiave del sistema
                    \end{itemize}
            \end{itemize}

        \item \textbf{}
        
    \end{itemize}


\section{Proprietà}
    L'obiettivo è sviluppare un sistema che consenta l'accesso remoto a servizi riservati agli utenti in possesso delle credenziali richieste.
    In questo contesto, diverse autorità rilasciano credenziali agli utenti, i quali le utilizzano per accedere a servizi qualificati, ossia servizi limitati ai possessori di specifiche credenziali.
    
    Per garantire l'efficacia e la sicurezza del sistema, è necessario considerare diverse proprietà fondamentali: confidenzialità, integrità, trasparenza ed efficienza.
    Queste proprietà non solo assicurano che il sistema funzioni correttamente, ma proteggono anche contro potenziali minacce e abusi.

    
    \subsection{Confidenzialità}
        \begin{itemize}
            \item \textbf{C1:} I dati sensibili degli utenti devono essere protetti, ovvero accessibili solo dalle parti autorizzate in modo controllato
            
            \item \textbf{C2:} Garantire la privacy delle comunicazioni tra gli utenti e i server che offrono i servizi, in modo che non sia possibile intercettare o rubare le credenziali trasmesse

            \item \textbf{C3:} Garantire la privacy delle comunicazioni tra gli utenti e le autorità di rilascio delle credenziali, in modo che non sia possibile carpire dati personali relativi agli utenti
            
            \item \textbf{C4:} Tutelare gli utenti contro i malintenzionati, di modo che questi ultimi non possano utilizzare l'identità di altri utenti, accedendo ai servizi sfruttando la loro CIE

            \item \textbf{C5:} Garantire che un utente non sia in grado di richiedere una CIE valida sfruttando informazioni non correlate alla sua persona
        \end{itemize}


    \subsection{Integrità}
        \begin{itemize}
                \item \textbf{I1:} Preservare l'integrità delle credenziali rilasciate, garantendo che non possano essere alterate o falsificate
            
            \item \textbf{I2:} Il sistema deve garantire che le credenziali esibite dagli utenti siano verificabili e autentiche, ovvero rilasciate da autorità riconosciute come affidabili
            
            \item \textbf{I3:} Verificare che le credenziali fornite dall'utente siano valide e corrispondano alle informazioni contenute nella CIE

            \item \textbf{I4:} La CIE restituisce una firma del messaggio solo se il PIN fornito corrisponde a quello memorizzato, garantendo che solo l'utente legittimo possa ottenere una firma valida

            \item \textbf{I5:} Verificare che la CIE sia riconducibile all'utente che la utilizza
            
            \item \textbf{I6:} Se un servizio è rivolto ad un individuo in possesso di due credenziali allora non deve essere possibile per due utenti, ciascuno avente una sola credenziale, combinare le loro informazioni per accedere al servizio
        \end{itemize}


    \subsection{Trasparenza}
        \begin{itemize}
            \item \textbf{T1:} Il sistema di autenticazione tramite CIE dovrebbe basarsi su algoritmi noti e verificabili, di modo da permettere una verifica di essi da parte degli utenti, rendendo arduo il processo di manomissione da parte di un malintenzionato, senza che quest'ultimo sia scoperto  
        
            \item \textbf{T2:} Il processo di rilascio delle credenziali deve essere trasparente e verificabile da tutte le parti coinvolte, riducendo la dipendenza da terze parti fidate così da ridurre il rischio di abuso   
            
            \item \textbf{T3:} Assicurarsi che le politiche di accesso ai servizi siano chiare e comprensibili, consentendo agli utenti di sapere quali credenziali siano necessarie per accedere agli stessi, limitando così la possibilità di accessi non autorizzati

            \item \textbf{T4:} La fiducia in specifiche parti deve essere giustificata da motivazioni concrete, come la possibilità di rilevare e punire abusi o il preservare la buona reputazione
        \end{itemize}


    \subsection{Efficienza}
        \begin{itemize}
            \item \textbf{E1:} Il sistema deve consentire ai server di controllare l'accesso utilizzando politiche avanzate e flessibili, permettendo di definire criteri di accesso complessi e specifici. Questo include la possibilità di esprimere condizioni logiche avanzate (come AND, OR, NOT) basate sulle credenziali degli utenti
            
            \item \textbf{E2:} Garantire che il processo di rilascio e verifica delle credenziali sia eseguito in modo rapido ed efficiente, riducendo al minimo i tempi di attesa e l'impatto sulle prestazioni del sistema
            
            \item \textbf{E3:} Assicurare che l'ottenimento delle credenziali tramite CIE avvenga in modo semplice e veloce, facilitando l'accesso remoto ai servizi digitali

            \item \textbf{E4:} Progettare il sistema in modo che possa gestire un grande numero di utenti e richieste contemporaneamente senza degradare le performance

            \item \textbf{E5:} Assicurarsi che il sistema sia compatibile con vari dispositivi e piattaforme, inclusi dispositivi mobili e desktop, per massimizzare l'accessibilità
        \end{itemize}