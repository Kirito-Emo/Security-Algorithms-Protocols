\chapter{WORK PACKAGE 2}
In questo capitolo concentreremo la nostra attenzione nel presentare una soluzione che risponde al modello identificato nel \textit{Work Package 1 (WP1)}.
L’obiettivo è quello di proporre un sistema che riesca a raggiungere un ragionevole compromesso tra efficienza, trasparenza, confidenzialità e sicurezza.

\noindent Concentreremo la nostra attenzione sui seguenti problemi chiave:
    \begin{itemize}
        \item Richiesta e ottenimento della CIE
        
        \item Richiesta e ottenimento delle credenziali necessarie per l'accesso ai servizi specifici
        
        \item Autenticazione e accesso ai servizi qualificati
    
        \item Protezione della privacy e dell'integrità delle credenziali
    
        \item Minimizzazione del coinvolgimento di terze parti fidate
    \end{itemize}


    \section{Panoramica Generale di Funzionamento}
        Innanzitutto, a valle di una richiesta apposita, viene emessa una CIE con il PIN associato per ogni utente.
        Ciascuno di essi, tramite l'utilizzo della CIE, può richiedere delle credenziali ad un'autorità di rilascio delle stesse.
        Esistono varie autorità che rilasciano credenziali agli utenti e, una volta entratone in possesso, è possibile utilizzarle per identificarsi e accedere ai servizi qualificati.
        È consentito, all’utente, l'accesso ai servizi qualificati solo se le credenziali soddisfano i requisiti di accesso imposti dal servizio stesso.


    \section{Generazione e Utilizzo delle Credenziali}
        In questa sezione verrà sviluppato un protocollo per la generazione e l'utilizzo di un sistema di identificazione tramite credenziali per l'accesso a servizi digitali qualificati.
        Questo processo coinvolge i seguenti attori chiave: l'utente, l'autorità di rilascio delle credenziali e il server del servizio.
        I partecipanti interagiscono con l'autorità per la richiesta e la generazione delle credenziali e con il server del servizio per usufruire dei servizi offerti.
        Si inizierà descrivendo come un utente può richiedere una credenziale all'autorità, chiarendo anche il processo di rilascio telematico della stessa.
        Successivamente, verrà delineato il contenuto e la struttura delle credenziali, affinché possano essere utilizzate come strumento digitale per l'identificazione e l'accesso ai servizi, mantenendo la massima riservatezza delle informazioni contenute.
        Infine, analizzeremo il processo di identificazione e accesso a un servizio digitale.
                
        \subsection{Supposizioni}
            Il funzionamento del protocollo proposto si basa sulle seguenti supposizioni:

            \begin{itemize}
                \item L'utente ha ottenuto la CIE e il PIN associato

                \item L'autorità di rilascio è fidata, emette credenziali corrette e rispetta il protocollo descritto

                \item Le informazioni necessarie alla verifica delle credenziali sono pubbliche, incluse gli algoritmi e i protocolli utilizzati

                \item L'autorità di rilascio possiede certificati validi per firmare le credenziali e per la comunicazione TLS con l'utente
            \end{itemize}

            \noindent Si procede adesso con l’analisi dettagliata di quanto delineato fino a questo momento.


        \subsection{Processo di Richiesta e Rilascio delle Credenziali}
            
            \subsubsection{Generazione di una coppia di chiavi}
                \begin{itemize}
                    \item Ogni utente genera una coppia di chiavi pubblica/privata
    
                    \item La chiave pubblica $pk_{utente}$ è generata in accordo all'algoritmo Gen(1$^n$)
                    
                    \item La chiave pubblica dell'utente è registrata presso le autorità competenti
                \end{itemize}
            
            \subsubsection{Richiesta delle Credenziali}
                \begin{itemize}
                    \item  L'utente invia la sua chiave pubblica e la CIE alla CA
                    
                    \item La CA verifica l'identità dell'utente e, se verificata, emette una credenziale firmata digitalmente (X.509) contenente informazioni specifiche
    
                    \item Ogni certificato X.509 emesso contiene un numero di serie univoco che identifica in modo univoco la credenziale associata all'utente
                \end{itemize}
    
            \subsubsection{Emissione della Credenziale}
                \begin{itemize}
                    \item L'autorità firma digitalmente la credenziale utilizzando la propria chiave privata, creando così un documento X.509 che attesta le informazioni specifiche dell'utente
                \end{itemize}
    
    
        \subsection{Processo di Identificazione e Accesso ai Servizi}
            Quando un utente presenta una credenziale per accedere a un servizio, il servizio deve verificare che l'utente possieda effettivamente la chiave privata corrispondente alla chiave pubblica inclusa nel documento X.509.
    
            \subsubsection{Verifica delle Credenziali}
                \begin{itemize}
                    \item Il servizio decodifica la credenziale X.509 e verifica la firma digitale dell'autorità
    
                    \item Controlla che le informazioni contenute nella credenziale soddisfino i requisiti di accesso
    
                    \item Verifica che il numero di serie della credenziale non sia incluso nell'elenco delle revocazioni più aggiornato (CRL)
                \end{itemize}
    
            \subsubsection{Autenticazione dell'Utente}
                \begin{itemize}
                    \item Quando l'utente desidera accedere a un servizio, utilizza la firma di Schnorr per autenticarsi al server
    
                    \item La firma di Schnorr permette di dimostrare la conoscenza della chiave privata senza rivelarla, aumentando così la sicurezza
                \end{itemize}
    
            \subsubsection{Accesso ai Servizi}
                \begin{itemize}
                    \item Se le credenziali soddisfano i requisiti di accesso imposti dal servizio e non sono state revocate, l'utente ottiene l'accesso al servizio richiesto
                    
                    \item Il server registra l'accesso dell'utente in modo sicuro e trasparente, mantenendo un log degli accessi per future verifiche
                \end{itemize}
    
    
        \subsection{Politiche di Revoca e Aggiornamento della Certificate Revocation List (CRL)}
            \begin{itemize}
                \item Ogni certificato X.509 emesso contiene un numero di serie univoco
    
                \item Se un utente sospetta che la sua chiave privata sia stata compromessa o persa, può richiedere alla CA la revoca del certificato corrispondente
    
                \item La CA mantiene un database dei numeri di serie dei certificati emessi e revocati
    
                \item Quotidianamente, la CA genera una Certificate Revocation List (CRL) contenente i numeri di serie dei certificati revocati e la firma digitale della CA
    
                \item La CRL include anche un campo CRL Distribution Point che indica l'URL da cui è possibile ottenere l'ultima versione della CRL
    
                \item I server dei servizi verificano la validità delle credenziali confrontando il numero di serie della credenziale con la CRL più aggiornata
            \end{itemize}
    
    
    \section{Minimizzazione del Coinvolgimento di Terze Parti Fidate}
        Per minimizzare il coinvolgimento di terze parti fidate durante gli accessi ai servizi:
        
        \begin{itemize}
            \item Le credenziali sono emesse una sola volta come documenti X.509 e non richiedono la verifica continua da parte delle autorità di rilascio durante ogni accesso
            
            \item I server dei servizi qualificati verificano autonomamente i documenti X.509 delle credenziali utilizzando le informazioni pubbliche rese disponibili dalle autorità di rilascio
            
            \item L'utilizzo di firme digitali e connessioni TLS garantisce la sicurezza e l'integrità delle comunicazioni
        \end{itemize}
    
    
    \section{Protezione della Privacy e Integrità delle Credenziali}
        Per proteggere la privacy degli utenti e l'integrità delle credenziali:
        \begin{itemize}
            \item Le credenziali sono emesse come documenti X.509 e contengono solo le informazioni strettamente necessarie per l'accesso ai servizi, riducendo l'esposizione di dati personali
                
             \item L'identità dell'utente non è rivelata al server del servizio qualificato attraverso i documenti X.509 delle credenziali, a meno che non sia strettamente necessario
                
             \item Le credenziali sono firmate digitalmente dalle autorità di rilascio, garantendo che non possano essere alterate o falsificate
        \end{itemize}
    
    
    \section{Conclusione}
        La soluzione proposta mira a raggiungere un compromesso tra efficienza, trasparenza, confidenzialità e sicurezza.
        Il sistema descrive dettagliatamente le azioni delle parti oneste coinvolte, garantendo che le credenziali siano emesse e verificate in modo sicuro e trasparente.
        L'uso di certificati X.509 firmati digitalmente dalle autorità di certificazione (CA) assicura che le credenziali siano autentiche e non alterabili.
        Inoltre, l'adozione della firma di Schnorr per l'autenticazione al server minimizza il coinvolgimento di terze parti fidate durante l'accesso ai servizi, proteggendo la privacy e l'integrità degli utenti.
        Questa architettura bilancia le esigenze di sicurezza e confidenzialità con la necessità di un sistema efficiente e utilizzabile, riducendo al minimo i rischi associati agli attacchi su larga scala e garantendo un alto livello di affidabilità e trasparenza.
    

    \section{Proposta di Soluzione}
        %Inserire testo

        \subsection{Possibile vulnerabilità}
            % Inserire testo

        \subsection{Proposta \#2 di Soluzione}
            % Inserire nuove info su 


    \section{Specifiche degli algoritmi utilizzati}
        % Inserire testo

        \subsection{Gen}
            % Inserire testo