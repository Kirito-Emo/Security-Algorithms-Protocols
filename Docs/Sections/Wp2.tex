\chapter{WORK PACKAGE 2}
In questo capitolo concentreremo la nostra attenzione nel presentare una soluzione che risponde al modello identificato nel \textit{Work Package 1 (WP1)}. L’obiettivo è quello di proporre un sistema che riesca a raggiungere un ragionevole compromesso tra efficienza, trasparenza, confidenzialità e sicurezza.

\noindent Concentreremo la nostra attenzione sui seguenti problemi chiave:
    \begin{itemize}
        \item Richiesta e ottenimento della CIE
        
        \item Richiesta e ottenimento delle credenziali necessarie per l'accesso ai servizi specifici
        
        \item Autenticazione e accesso ai servizi qualificati
    
        \item Protezione della privacy e dell'integrità delle credenziali
    
        \item Minimizzazione del coinvolgimento di terze parti fidate
    \end{itemize}


    \section{Panoramica Generale di Funzionamento}
        Innanzitutto, a valle di una richiesta apposita, viene emessa una CIE con il PIN associato per ogni utente.
        Ciascuno di essi, tramite l'utilizzo della CIE, può richiedere delle credenziali ad un'autorità di rilascio delle stesse.
        Esistono varie autorità che rilasciano credenziali agli utenti e, una volta entratone in possesso, è possibile utilizzarle per identificarsi e accedere ai servizi qualificati.
        È consentito, all’utente, l'accesso ai servizi qualificati solo se le credenziali soddisfano i requisiti di accesso imposti dal servizio stesso.


    \section{Generazione e Utilizzo delle Credenziali}
        In questa sezione verrà sviluppato un protocollo per la generazione e l'utilizzo di un sistema di identificazione tramite credenziali per l'accesso a servizi digitali qualificati.
        Questo processo coinvolge i seguenti attori chiave: l'utente, l'autorità di rilascio delle credenziali e il server del servizio.
        I partecipanti interagiscono con l'autorità per la richiesta e la generazione delle credenziali e con il server del servizio per usufruire dei servizi offerti.
        Si inizierà descrivendo come un utente può richiedere una credenziale all'autorità, chiarendo anche il processo di rilascio telematico della stessa.
        Successivamente, verrà delineato il contenuto e la struttura delle credenziali, affinché possano essere utilizzate come strumento digitale per l'identificazione e l'accesso ai servizi, mantenendo la massima riservatezza delle informazioni contenute.
        Infine, analizzeremo il processo di identificazione e accesso a un servizio digitale.
                
        \subsection{Supposizioni}
            Il funzionamento del protocollo proposto si basa sulle seguenti supposizioni:

            \begin{itemize}
                \item L'utente ha ottenuto la CIE e il PIN associato

                \item L'autorità di rilascio è fidata, emette credenziali corrette e rispetta il protocollo descritto

                \item Le informazioni necessarie alla verifica delle credenziali sono pubbliche, incluse gli algoritmi e i protocolli utilizzati

                \item L'autorità di rilascio possiede certificati validi per firmare le credenziali e per la comunicazione TLS con l'utente
            \end{itemize}

    \noindent Si procede adesso con l’analisi dettagliata di quanto delineato fino a questo momento.


    \section{Processo di Richiesta e Rilascio delle Credenziali}
    
        \subsection{Generazione di una coppia di chiavi}
            \begin{itemize}
                \item Ogni utente genera una coppia di chiavi pubblica/privata (se non l'ha già fatto durante la fase di registrazione della CIE)

                \item Un utente genera la chiave pubblica $pk_{utente}$ in accordo all'algoritmo Gen(1$^n$)
                
                \item La chiave pubblica dell'utente è registrata presso le autorità competenti
            \end{itemize}
    
        \subsection{Richiesta delle Credenziali}
            \begin{itemize}
                \item  L'utente firma digitalmente la richiesta di credenziali con la propria chiave privata
                
                \item L'autorità competente verifica la firma utilizzando la chiave pubblica registrata
            \end{itemize}

        \subsection{Emissione del token di credenziale}
            \begin{itemize}
                \item Il token di credenziale contiene un campo aggiuntivo per la chiave pubblica dell'utente
                
                \item L'autorità firma il token di credenziale includendo la chiave pubblica dell'utente
            \end{itemize}


    \section{Processo di Identificazione e Accesso ai Servizi}
        Quando un utente presenta il token di credenziale per accedere a un servizio, il servizio deve verificare che l'utente possieda effettivamente la chiave privata corrispondente alla chiave pubblica inclusa nel token.

        \subsection{Verifica del Token}
            \begin{itemize}
                \item Il servizio decodifica il token e verifica la firma digitale dell'autorità

                \item Controlla che l'attributo del token corrisponda ai requisiti di accesso
            \end{itemize}

        \subsection{Autenticazione dell'Utente}
            \begin{itemize}
                \item Il servizio genera una sfida (\textit{challenge}) crittografica

                \item L'utente deve firmare la sfida con la propria chiave privata

                \item Il servizio verifica la firma utilizzando la chiave pubblica inclusa nel token
            \end{itemize}
    
        \subsection{Accesso ai Servizi}
            \begin{itemize}
                \item Se le credenziali soddisfano i requisiti di accesso imposti dal servizio, l'utente ottiene l'accesso al servizio richiesto.
                
                \item Il server registra l'accesso dell'utente in modo sicuro e trasparente, mantenendo un log degli accessi per future verifiche.
            \end{itemize}


    \section{Minimizzazione del Coinvolgimento di Terze Parti Fidate}
        Per minimizzare il coinvolgimento di terze parti fidate durante gli accessi ai servizi:
    
        \begin{itemize}
            \item Le credenziali sono emesse una sola volta e non richiedono la verifica continua da parte delle autorità di rilascio durante ogni accesso
        
            \item I server dei servizi qualificati verificano autonomamente le credenziali utilizzando le informazioni pubbliche rese disponibili dalle autorità di rilascio
        
            \item L'utilizzo di firme digitali e connessioni TLS garantisce la sicurezza e l'integrità delle comunicazioni tra l'utente e i server dei servizi
        \end{itemize}

    
    \section{Protezione della Privacy e dell'Integrità delle Credenziali}
        Per proteggere la privacy degli utenti e l'integrità delle credenziali:
        \begin{itemize}
            \item Le credenziali contengono solo le informazioni strettamente necessarie per l'accesso ai servizi, minimizzando l'esposizione di dati personali
            
            \item L'identità dell'utente non è rivelata al server del servizio qualificato, a meno che non sia strettamente necessario
            
            \item Le credenziali sono firmate digitalmente dalle autorità di rilascio, garantendo che non possano essere alterate o falsificate
        \end{itemize}


    \section{Conclusione}
        La soluzione proposta mira a raggiungere un compromesso tra efficienza, trasparenza, confidenzialità e sicurezza. Descrive dettagliatamente le azioni delle parti oneste coinvolte nel sistema, garantendo che le credenziali siano emesse e verificate in modo sicuro e trasparente, minimizzando il coinvolgimento di terze parti fidate e proteggendo la privacy e l'integrità degli utenti.